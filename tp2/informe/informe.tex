\documentclass[final,inline,a4paper,narroweqnarray]{ieee}
% In order to use the figure-defining commands in ieeefig.sty...
\usepackage{ieeefig}
% To use utf8 encoding
\usepackage[T1]{fontenc}
\usepackage[utf8]{inputenc}
\usepackage[spanish]{babel}
% to place figures
\usepackage{float}
\usepackage{graphicx}
\usepackage{amsmath}
% dont use geometry package
%\usepackage{geometry}

\usepackage{header}

\begin{document}

%----------------------------------------------------------------------
% Title Information, Abstract and Keywords
%----------------------------------------------------------------------
\title[TP N$^o$ 2: Rutas en Internet]{%
Trabajo Práctico N$^o$ 2: Rutas en Internet}

% format author this way for journal articles.
\author[Barrios, Benegas, Caravario, Rodriguez]{%
	Leandro Ezequiel Barrios,
	\and
	Gonzalo Benegas,
	\and
	Martin Caravario,
	\and
	Pedro Rodriguez
}

% make the title
\maketitle

% do the abstract
\begin{abstract}

En el presente Trabajo Práctico nos propondremos experimentar con herramientas y
técnicas frecuentemente utilizadas a nivel de red. En particular, nos
centraremos en la utilización de \texttt{traceroute} como herramienta para medir
tiempos y \texttt{RTTS}.

\end{abstract}

% start the main text ...

%----------------------------------------------------------------------
% SECTION I: Introduction
%----------------------------------------------------------------------
\section{Introducción}

Los objetivos del presente Trabajo Práctico son múltipes: por un lado, entender
los protocolos involucrados en cada comunicación o intercambio de datos entre un
usuario y un servidor, estando ambos conectados a Internet. Vimos en clase que
estos protocolos son tanto de nivel de capa de transporte (TCP, UDP) como de
aplicación (DNS, SMTP). Por otro lado, para afianzar nuestros conocimientos,
implementaremos una herramienta que cumpla la misma función que
\texttt{traceroute} por nuestra cuenta. La función de esta herramienta es la de
detectar, a partir del nombre de un dominio (host), cada uno de los nodos
intermedios por los que pasa el paquete antes de llegar a la dirección IP
correspondiente a dicho host. En este TP, analizaremos los RTT de paquetes
enviados a distintas universidades del mundo.


%----------------------------------------------------------------------
% SECTION II: Caracterizando Rutas
%----------------------------------------------------------------------
\section{Caracterizando Rutas}

  %----------------------------------------------------------------------
  %  SUBSECTION III: Implementación de tool para hacer traceroute
  %----------------------------------------------------------------------
  \subsection{Implementación de tool para hacer traceroute}

  Implementamos la herramienta \texttt{traceroute2} en python, usando la
  librería \emph{scapy}. Para esto, armamos varios paquetes echo-request,
  inicianizándolos con un TTL entre 1 y alguna cantidad de hops variable, que
  nos permita llegar al destino (en general, 30 o 40 hops son más que
  suficientes). A cada paquete ICMP que es respondido, le corresponderá un RTT
  diferente y, los paquetes que no son respondidos después de un tiempo de
  timeout determinado, asumimos que fueron recibidos y no respondidos o que se
  perdieron en la red, marcando que el RTT y resto de la información
  correspondiente a dicho paquete con un $*$. \\ Una herramienta extra que
  implementamos para facilitar la experimentación fue una función que llamamos
  DNSInverso, que para cada router nos permite reconocer su nombre. De esta
  forma podemos detectar, para cada máquina de la que obtenemos respuesta en
  algún hop, a qué organización pertenece (si es que pertenece a alguna). Y, en
  particular, si pertenece a la universidad a la que estamos haciendo el
  traceroute.

  A partir de los valores de RTT obtenidos para cada nodo intermedio de la ruta
  rastreada por \texttt{traceroute2}, graficamos el RTT observado entre dicho
  nodo y el nodo origen.

  %----------------------------------------------------------------------
  %  SUBSECTION III: Adaptación de la tool
  %----------------------------------------------------------------------
  \subsection{Adaptación de la tool}

  Después de haber experimentado con la tool original, enviando paquetes echo-
  request a varias universidades alrededor del mundo, adaptamos dicha
  \emph{tool} para que, una vez terminada la búsqueda, calcule el \emph{valor
  standard}  o \emph{valor Z (ZRTT) } en cada salto con respecto a los saltos
  realizados en la ruta global de la siguiente manera: si $RTT_i$ es el RTT
  medido para el salto $i$, se define
  \[
  ZRTT_i = \dfrac{RTT_i - \overline{RTT}}{SRTT}
  \]
  donde $\overline{RTT}$ y $SRTT$ son el promedio y el desvío estandard de una
  lista que contiene las diferencias de RTT entre cada hop y el hop que lo
  antecede en la ruta que realiza el paquete.

  Luego, para cada hop $i$ tendremos un valor $|ZRTT_i|$. Por cómo definimos a
  $ZRTT_i$, se observa que a mayor $valor Z$ para un hop, mayor será la
  variación del tiempo que se tarda en atravesar dicho hop en comparación con el
  promedio del tiempo que se tarda en atravesar cada hop de la ruta.

  A partir de esta adaptación esperamos introducir en los gráficos de RTT =
  f(número de hop) que habíamos hecho con la $tool$ original la nueva
  información obtenida sobre cada hop. Esperamos que esto nos permita detectar
  aquellos hops que corresponden a enlaces submarinos. En particular, si el hop
  $i$ tiene un $|ZRTT_i|$ mucho mayor que el resto, sería un indicator de que en
  el hop $i$ el paquete atraviesa un enlace submarino.

%----------------------------------------------------------------------
% SECTION III: Traceroute a universidades del mundo
%----------------------------------------------------------------------
\section{Experimentación: Traceroute a universidades del mundo}

A partir de la herramienta \texttt{traceroute2} y tomando como IP origen la de
la PC de uno de los integrantes del grupo, realizamos rastreos para distintas
universidades alrededor del mundo. Cada una de estas universidades las
identificamos a partir de su URL correspondiente. En el proceso de la
experimentación, introdujimos los datos obtenidos para cada universidad en un
graficador, que dada la lista de direcciones IP recorridas, las localiza
geográficamente en un mapa del planeta Tierra, y las une mediante una línea.

Las universidades que terminamos eligiendo fueron: en Noruega (Oslo) a
<<www.uio.no>>, en Japón (Tokyo) a <<www.u-tokyo.ac.jp>>, en Tanzania (Dar es
Salaam) a <<www.udsm.ac.tz>> y en Nueva Zelanda (Auckland) a
<<www.auckland.ac.nz>>.

De esta forma, intentamos expresar de la forma más gráfica posible el camino a
escala global que debió efectuar cada paquete para llegar desde nuestra PC hasta
algún servidor de dicha universidad, ubicada a decenas de miles de kilómetros de
nosotros.

Uno de los problemas con que nos encontramos encontrar universidades para las
cuáles los trayectos efectuados por los paquetes tuvieran distintas
características topológicas (por ejemplo que los paquetes atraviesen enlaces
submarinos o que las universidades queden en distintos continentes), a fin de
poder realizar posteriormente un análisis más completo e interesante de los RTTs
y ZRTTs correspondientes a cada trayectoria.

Dos de los problemas con los que nos encontramos durante la experimentación
fueron:
\begin{itemize}

  \item Tuvimos problemas con el envío de paquetes a través de conexión wi-fi.
  Es por esto que tuvimos que realizar todas las mediciones con una misma
  computadora, conectada a internet a través de una conexión cableada.

  \item Encontramos muchas universidades, como por ejemplo la de Nairobi
  (www.uombi.ac.ke), para las cuáles el camino graficado en el mapa terminaba en
  EEUU, en lugar de de en África, como esperábamos. Para estos casos, supusimos
  que el problema no era de un bug en nuestra implementación, sino en que dicha
  página web estaba hosteada en un servidor localizado en EEUU. Una posible
  razón por la cuál dicha universidad querría hacer esto podría ser, por
  ejemplo, que el costo de hostear un servidor en Kenia es mayor que en EEUU.

  \item En todos los casos analizados, hubieron muchos nodos intermedios que no
  contestaban nuestro \emph{ICMP Request}, con lo cuál habían muchos \emph{hops}
  que no éramos capaces de localizar en el mapa, lo cuál restaba precisión a
  nuestra representación de la trayectoria de los paquetes.

  \item Hubieron algunas universidades para las cuáles (por ejemplo, al enviar
  40 paquetes a www.udsm.ac.tz) había un primer router localizado en dicha
  universidad nos contestaba, pero después de obtener esta respuesta, dejábamos
  de obtener respuesta. De esta forma, no logramos obtener una respuesta de la
  IP destino. Creemos quue este último router que nos contestaba estaba actuando
  como firewall de la red interna de dicha universidad.

\end{itemize}

\subsection{Gráficos y análisis}

\subsubsection{Universidad de Auckland}

\begin{figure}[ht]\begin{center}
   \includegraphics[width=0.5\textwidth]{%
    imagenes/auckland.png}
    \caption{Auckland}
    \label{auckland-rtt-zrtt}
\end{center}\end{figure}

\begin{figure}[ht]\begin{center}
   \includegraphics[width=0.5\textwidth]{%
    imagenes/auckland-mapa.png}
    \caption{Auckland - mapa}
    \label{auckland-mapa}
\end{center}\end{figure}

\subsubsection{Universidad de Tanzania}
\begin{figure}[ht]\begin{center}
   \includegraphics[width=0.5\textwidth]{%
    imagenes/tanzania.png}
    \caption{Tanzania}
    \label{tanzania-rtt-zrtt}
\end{center}\end{figure}

\begin{figure}[ht]\begin{center}
   \includegraphics[width=0.5\textwidth]{%
    imagenes/tanzania-mapa.png}
    \caption{Tanzania - mapa}
    \label{tanzania-mapa}
\end{center}\end{figure}

\subsubsection{Universidad de Tokyo}
\begin{figure}[ht]\begin{center}
   \includegraphics[width=0.5\textwidth]{%
    imagenes/japon.png}
    \caption{Tokyo}
    \label{tokyo-rtt-zrtt}
\end{center}\end{figure}

\begin{figure}[ht]\begin{center}
   \includegraphics[width=0.5\textwidth]{%
    imagenes/japon-mapa.png}
    \caption{Tokyo - mapa}
    \label{tokyo-mapa}
\end{center}\end{figure}

\subsubsection{Universidad de Oslo}
\begin{figure}[ht]\begin{center}
   \includegraphics[width=0.5\textwidth]{%
    imagenes/noruega-oslo.png}
    \caption{Noruega}
    \label{noruega-rtt-zrtt}
\end{center}\end{figure}

\begin{figure}[ht]\begin{center}
   \includegraphics[width=0.5\textwidth]{%
    imagenes/noruega-oslo-mapa.png}
    \caption{Noruega (Oslo) - mapa}
    \label{noruega-mapa}
\end{center}\end{figure}

\subsubsection{Análisis general}

Lo primero que nos llamó la atención de estos histogramas fue que los RTTs
observados entre dos hops consecutivos no siempre son crecientes. Esto en
principio no tiene sentido, pues cualquier paquete, para llegar al hop $i$,
necesariamente debe pasar antes por el hop $i-1$. Luego, una posible explicación
para esto es que alguno/s router/s de la ruta estén utilizando colas de
prioridad para los paquetes, y que en particular los ICMP (los que nosotros
estamos enviando) tengan una prioridad baja.

%----------------------------------------------------------------------
% SECTION IV: Contrastando con la realidad
%----------------------------------------------------------------------
\section{Contrastando con la realidad}

A continuación simulamos una conexión mediante sucesivos \emph{pings}
(\emph{EchoRequest} $+$ \emph{EchoReply}) a cada una de las universidades
analizadas, y estimamos el \emph{RTT} a dicho destino en base a la siguiente
ecuación:
\[
EstimatedRTT = \alpha * EstimatedRTT + (1 - \alpha) * SampleRTT
\]
Con la introducción de este parámetro $alpha$, lo que haremos es ir actualizando
el $EstimatedRTT$ en cada hop del recorrido del paquete en función del RTT que
hubo en el hop anterior y en el actual, dándoles un peso (importancia) a cada
uno.

Variamos la cantidad de paquetes enviados ($n$) y $alpha$ para obtener los
parámetros más adecuados para analizar los resultados y contrastarlos con los
resultados obtenidos en la sección anterior. Esta elección previa de parámetros
se realizó designando como host a la Universidad de Auckland.

\begin{figure}[ht]\begin{center}
   \includegraphics[width=0.5\textwidth]{imagenes/auckland-estimate-alpha-08.png}
    \caption{RTT estimado para distintos \texttt{n}, con \texttt{alpha}=0.8}
    \label{auckland-estimate-alpha-08}
\end{center}\end{figure}

Es interesante el valor estimado del RTT usando distintos valores de $n$. Dada la fórmula de estimación, el tener más mediciones debería moderar el efecto de outliers, resultando en una estimación más uniforme.
En efecto, en \fig{auckland-estimate-alpha-08} se puede observar una variabilidad grande entre haber medido 10, 50 y 100 paquetes, pero un resultado mucho más estable tras haber medido 500 y 1000 paquetes.

\begin{figure}[ht]\begin{center}
   \includegraphics[width=0.5\textwidth]{imagenes/auckland-estimate-n-1000.png}
    \caption{RTT estimado para distintos \texttt{alpha}, con \texttt{n}=1000}
    \label{fig:auckland-estimate-n-1000}
\end{center}\end{figure}

Cuando $alpha$ está cercano a 0, estamos estimando el RTT hasta el host solo en función de la última medición. A medida que $alpha$ se acerque a 1, le estamos dando más importancia a nuestro conocimiento previo sobre la congestión en la red y el valor del RTT. Parece sensato elegir un $alpha$ bastante grande para aprovechar la información previa y hacer una estimación estadística. Un $alpha$ chico resultaría en una estimación del RTT muy volátil, afectada por la presencia de $outliers$.
De hecho, el estandar TCP define $alpha=0.875$\footnote{http://tools.ietf.org/html/rfc6298}.
Nuestros resultados, expuestos en \fig{auckland-estimate-n-1000}, muestran un RTT estimado más bajo para los valores de $alpha$ más altos. Sin embargo, la diferencia no parece demasiado grande en relación al RTT. Se estima que las diferencias se apreciarían más en redes muy congestionadas.

Por último, se nos pide que analizemos el eventual $throughput$ punta a punta
que se deriva de la \emph{ecuación de Mathis}:
\[
Throughput \leq \frac{MSS}{RTT*\sqrt{p}}
\]

donde $p$ es la probabilidad estimada de pérdida de paquetes, definida de la siguiente manera:
\[
p = 1 - \dfrac{\#Echo reply}{\#Echo request}
\]
$MSS$ es el $Maximum Window Size$,

\begin{tabular}{| p{1.6cm} | p{1.2cm} | p{1.4cm} | p{1.2cm} | p{1.7cm} |}
  \hline
  Universidad & RTT inicial & Estimated RTT & $p$ & Throughput de Mathis \\
  \hline
  Auckland & 399.84 & 423.81 & 0.005 & 48.72 \\
  Oslo & 305.48 & 337.61 & 0.003 & 78.95 \\
  Tanzania & 468.56 & 495.57 & 0.004 & 46.58 \\
  Tokyo & 372.04 & 394.87 & 0.006 & 47.73 \\
  \hline
  \label{fig:mathis}
\end{tabular}

\end{document}
